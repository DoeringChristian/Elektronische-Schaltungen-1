% .:: Laden der LaTeX4EI Formelsammlungsvorlage
\documentclass[european]{latex4ei_sheet}

\title{Elektronische \\ Schaltungen \\ Cheatsheet}

\RequirePackage{latex4ei/latex4ei_unicode}

% -- begin custom TOC for circuits
\usepackage{tocloft}

\newcommand{\listcircuitname}{Liste der Schaltungen}
\newlistof{circuit}{cir}{}
\newcommand{\circuit}[2]{%
	\refstepcounter{circuit}
	\section{#1}
	\includegraphics[width=5em]{#2}
	\addcontentsline{cir}{blabla}
	{\parbox{\linewidth}{\includegraphics[width=\linewidth]{#2}\\#1\hfill}\ \\}}
\let\mylistofcircuit\listofcircuit
\renewcommand{\listofcircuit}{\subsection*{\listcircuitname}\vspace{-1.5em}\begin{multicols}{3}\mylistofcircuit\end{multicols}}
% -- end


% DOCUMENT_BEGIN ===============================================================
\begin{document}

\IfFileExists{git.id}{\input{git.id}}{}
\ifdefined\GitRevision\mydate{\GitNiceDate\ (git \GitRevision)}\fi

% Title (needs ./img/Logo.pdf)
\maketitle

\listofcircuit

\circuit{Stromspiegel mit Widerstand}{img/stromspiegel-mit-widerstand.pdf}
Annahme: Sättigung $\rightarrow V_{\ir GS1} = V_{\ir DD}-RI_{\ir IN}$\\
Überprüfe, ob Annahme korrekt\\
$\frac{I_{\ir out}}{I_{\ir in}} = \frac{(W/L)_2}{(W/L)_1}$
\circuit{Schalter (Transmission-Gate)}{img/transmission-gate.pdf}
$V_{\ir out} \approx V_{\ir in} \rightarrow$ Transistoren in Triodenbereich\\
\circuit{Transistor mit resistiver Last (Source an GND, immer +)}{img/transistor-mit-resistiver-last-1.pdf}
	%TODO ESB
	Verstärkung: $\frac{V_{\ir out}}{V_{\ir in}} = g_{\ir m1}(r_{\ir DS1}\parallel R_D)$\\
	$r_{\ir out} = r_{\ir DS1}\parallel R_D$\\
\circuit{Transistor mit resistiver Last ($V_{\ir in}$ an Source)}{img/transistor-mit-resistiver-last-2.pdf}	
	$A \approx g_mR_d$\\
	$r_\text{out} = r_{DS} \parallel R_D \approx R_D$\\
	$r_{\ir in} = R_D \frac{r_{\ir DS}\parallel \frac{1}{g_m}}{r_{\ir DS}\parallel R_D}$
	
\circuit{Transistor mit resistiver Last ($R_S$ an Source)}{img/transistor-mit-resistiver-last-3.pdf}	
	
	
\circuit{Sourceschaltung mit Stromquelle (- - -)}{img/sourceschaltung-mit-stromquelle.pdf}
%TODO ESB
Verstärkung: $\frac{V_{\ir out}}{V_{\ir in}} = -g_{\ir m1}r_{\ir DS1}$\\
$r_{\ir out} = r_{\ir DS1}$
\circuit{Sourceschaltung mit aktiver Last (- -)}{img/sourceschaltung-mit-aktiver-last.pdf}
%TODO ESB
Verstärkung: $\frac{V_{\ir out}}{V_{\ir in}} = -g_{\ir m1}(r_{\ir DS1}\parallel r_{\ir DS2})$\\
$r_{\ir out} = r_{\ir DS1}\parallel r_{\ir DS2}$
\circuit{Sourceschaltung mit Diodenlast [$V_{\ir in}$ am unteren Transistor (-)]}{img/sourceschaltung-mit-diodenlast-1.pdf}
	%TODO ESB
	Verstärkung: $\frac{V_{\ir out}}{V_{\ir in}} = -g_{\ir m1}\left( r_{\ir DS1}\parallel r_{\ir DS2}\parallel \frac{1}{g_{\ir m2}}\right)$\\
	$r_{\ir out} = r_{\ir DS1}\parallel r_{\ir DS2}\parallel \frac{1}{g_{\ir m2}}$
\circuit{Sourceschaltung mit Diodenlast [$V_{\ir in}$ am oberen Transistor]}{img/sourceschaltung-mit-diodenlast-2.pdf}

\circuit{Drainschaltung mit aktiver Last [ohne $C_L$ ($\approx 1$)]}{img/drainschaltung-mit-aktiver-Last-1.pdf}
	Verstärkung: $\frac{V_{\ir out}}{V_{\ir in}} = \frac{g_{\ir m1}(r_{\ir DS1}\parallel r_{\ir DS2})}{1 + g_{\ir m1}(r_{\ir DS1}\parallel r_{\ir DS2})}$ (kann auch $<1$ sein)\\
	$r_{\ir out} = \frac{r_{\ir DS1}\parallel r_{\ir DS2}}{1 + g_{\ir m1}(r_{\ir DS1}\parallel r_{\ir DS2})}$
\circuit{Drainschaltung mit aktiver Last [mit $C_L$]}{img/drainschaltung-mit-aktiver-Last-2.pdf}
	%TODO ESB
	Verstärkung: $A(\omega) = \frac{g_{\ir m1}(r_{\ir DS1}\parallel r_{\ir DS2})}{1 + g_{\ir m1}(r_{\ir DS1}\parallel r_{\ir DS2})+\j\omega C_L(r_{\ir DS1}\parallel r_{\ir DS2})}$\\
	Näherung: $\frac{g_{\ir m1}}{g_{\ir m1}+\j\omega C_L}$\\
	Betrag: $\abs{A(\omega)} = \frac{1}{\sqrt{1+(\omega C_L / g_{\ir m1})^2}}$\\
	Phase: $\phi(A(\omega) = \arctan{(-\omega C_L / g_{\ir m1})}$\\

\circuit{Stromspiegel mit Stromquelle}{img/stromspiegel-mit-stromquelle.pdf}
\circuit{Differenzstufe mit resistiver Last}{img/differenzstufe-mit-resistiver-last.pdf}
%\circuit{Differenzstufe mit Stromspiegel-Last (pMOS)}{img/differenzstufe-mit-stromspiegel-last-1.pdf}
\circuit{Differenzstufe mit Stromspiegel-Last (nMOS)}{img/differenzstufe-mit-stromspiegel-last-2.pdf}
\circuit{Differenzstufe mit Diodenlast}{img/differenzstufe-mit-diodenlast.pdf}

\section{x1}

\section{x2}

\section{x3}

\circuit{Invertierender Verstärker}{img/invertierender-verstärker.pdf}
%TODO ESB
Verstärkung (ideal): $A = -\frac{R_2}{R_1}$\\
Verstärkung (real): $A = \frac{A\frac{R_2}{R_1+R_2}}{1+A\frac{R_1}{R_1+R_2}}$\\
Ausgangswiderstand: $r_{\ir out} = 0$\\
Eingangswiderstand: $r_{\ir in} = R_1$\\

\circuit{Nichtinvertierender Verstärker}{img/nichtinvertierender-verstärker.pdf}
\subsection{$V_{\ir ref} = 0$}
Verstärkung (ideal): $A = 1 + \frac{R_2}{R_1}$\\
Verstärkung (real): $A = \frac{A}{1+A\frac{R_1}{R_1+R_2}}$\\
Ausgangswiderstand: $r_{\ir out} = 0$\\
Eingangswiderstand: $r_{\ir in} = \infty$\\
\subsection{$V_{\ir ref} \neq 0$}
$A=\frac{V_{\ir out,max}-V_{\ir ref}}{V_{\ir in}-V_{\ir ref}}$


\circuit{Summierverstärker}{img/summierverstärker.pdf}
Ausgangsspannung: $V_{\ir out} = - \sum_{k=1}^N V{\ir ein,k}\frac{R_3}{R_k}$\\
Fall $k=2$: $-\left( \frac{V_{\ir in,1}}{R_1} + \frac{V_{\ir in,2}}{R_2}R_3\right)$\\ 
Verstärkung (real): $A = \frac{A}{1+A\frac{R_1}{R_1+R_2}}$\\
Ausgangswiderstand: $r_{\ir out} = 0$\\
Eingangswiderstand: $r_{\ir in} = \infty$\\

\circuit{Differenzverstärker}{img/differenzverstärker.pdf}
Ausgangsspannung: $V_{\ir out} = V_{\ir in,p}\frac{R_4}{R_3+R_4} \frac{R_1+R_2}{R_1} - \frac{R_2}{R_1} V_{\ir in,p}$\\
Fall $R_3 = R_4$ und $R_1 = R_2$: $V_{\ir out} = V_{\ir in,p} - V_{\ir in,p}$

\circuit{Miller Operationsverstärker}{img/miller-op.pdf}
%TODO ESB
Verstärkung Ausgangsstufe: $A = g_{\ir m5}(r_{\ir DS5}\parallel r_{\ir DS6})$
$C_1 = C_F(1+A) \approx C_FA$ \qquad $C_2 = C_F\left(1+\frac{1}{A}\right) \approx C_F$ \\
$C_{\ir out} = C_2 + C_L$\\
Grenzfrequenz: $f_{G1} = \frac{1}{2\pi C_1r_{\ir out,diff}}$ \quad $f_{G2} = \frac{1}{2\pi C_2r_{\ir out,ausgang}}$\\
$A_0 = g_{\ir m1}(r_{\ir DS2}\parallel r_{\ir DS4})g_{\ir m5}(r_{\ir DS5}\parallel r_{\ir DS6})$\\
Übertragungsfunktion: $A(\omega)=\frac{A_0}{\left(1+\j\frac{f}{f_1}\right)\left(1+\j\frac{f}{f_2}\right)}$\\


%TODO Spannungsfolger mit Lastwiderstand
%TODO normaler OP-Amp
%TODO evtl. Instrumentenverstärker


% DOCUMENT_END =================================================================
\end{document}