% .:: Laden der LaTeX4EI Formelsammlungsvorlage
\documentclass[european]{latex4ei_sheet}

\title{Elektronische \\ Schaltungen \\ Cheatsheet}

\RequirePackage{latex4ei/latex4ei_unicode}

% DOCUMENT_BEGIN ===============================================================
\begin{document}

\IfFileExists{git.id}{\input{git.id}}{}
\ifdefined\GitRevision\mydate{\GitNiceDate\ (git \GitRevision)}\fi

% Title (needs ./img/Logo.pdf)
\maketitle

\section{Stromspiegel mit Widerstand}
Annahme: Sättigung $\rightarrow V_{\ir GS1} = V_{\ir DD}-RI_{\ir IN}$\\
Überprüfe, ob Annahme korrekt\\
$\frac{I_{\ir out}}{I_{\ir in}} = \frac{(W/L)_2}{(W/L)_1}$
\section{Schalter (Transmission-Gate)}
$V_{\ir out} \approx V_{\ir in} \rightarrow$ Transistoren in Triodenbereich\\
\section{Transistor mit resistiver Last}
	\subsection{Source an GND (immer +)}
	%TODO ESB
	Verstärkung: $\frac{V_{\ir out}}{V_{\ir in}} = g_{\ir m1}(r_{\ir DS1}\parallel R_D)$\\
	$r_{\ir out} = r_{\ir DS1}\parallel R_D$\\
	\subsection{$V_{\ir in}$ an Source}
	$r_{\ir in} = R_D \frac{r_{\ir DS}\parallel \frac{1}{g_m}}{r_{\ir DS}\parallel R_D}$
	
	\subsection{$R_S$ an Source}
	
	
\section{Sourceschaltung mit Stromquelle (- - -)}
%TODO ESB
Verstärkung: $\frac{V_{\ir out}}{V_{\ir in}} = -g_{\ir m1}r_{\ir DS1}$\\
$r_{\ir out} = r_{\ir DS1}$
\section{Sourceschaltung mit aktiver Last (- -)}
%TODO ESB
Verstärkung: $\frac{V_{\ir out}}{V_{\ir in}} = -g_{\ir m1}(r_{\ir DS1}\parallel r_{\ir DS2})$\\
$r_{\ir out} = r_{\ir DS1}\parallel r_{\ir DS2}$
\section{Sourceschaltung mit Diodenlast}
	\subsection{$V_in$ am unteren Transistor (-)}
	%TODO ESB
	Verstärkung: $\frac{V_{\ir out}}{V_{\ir in}} = -g_{\ir m1}\left( r_{\ir DS1}\parallel r_{\ir DS2}\parallel \frac{1}{g_{\ir m2}}\right)$\\
	$r_{\ir out} = r_{\ir DS1}\parallel r_{\ir DS2}\parallel \frac{1}{g_{\ir m2}}$
	\subsection{$V_in$ am oberen Transistor}

\section{Drainschaltung mit aktiver Last}
	\subsection{ohne Lastkapazität ($\approx 1$)}
	Verstärkung: $\frac{V_{\ir out}}{V_{\ir in}} = \frac{g_{\ir m1}(r_{\ir DS1}\parallel r_{\ir DS2})}{1 + g_{\ir m1}(r_{\ir DS1}\parallel r_{\ir DS2})}$ (kann auch $<1$ sein)\\
	$r_{\ir out} = \frac{r_{\ir DS1}\parallel r_{\ir DS2}}{1 + g_{\ir m1}(r_{\ir DS1}\parallel r_{\ir DS2})}$
	\subsection{mit Lastkapazität}
	%TODO ESB
	Verstärkung: $A(\omega) = \frac{g_{\ir m1}(r_{\ir DS1}\parallel r_{\ir DS2})}{1 + g_{\ir m1}(r_{\ir DS1}\parallel r_{\ir DS2})+\j\omega C_L(r_{\ir DS1}\parallel r_{\ir DS2})}$\\
	Näherung: $\frac{g_{\ir m1}}{g_{\ir m1}+\j\omega C_L}$\\
	Betrag: $\abs{A(\omega)} = \frac{1}{\sqrt{1+(\omega C_L / g_{\ir m1})^2}}$\\
	Phase: $\phi(A(\omega) = \arctan{(-\omega C_L / g_{\ir m1})}$\\

\section{Gateschaltung mit resistiver Last}
- Drainschaltung mit aktiver last
-Transistor mit resistiver last
-sourceschaltung mit diodenlast

\section{Stromspiegel mit Stromquelle}

\section{Differenzstufe mit resistiver Last}

\section{Differenzstufe mit Stromspiegel-Last}
	\subsection{pMOS-Differenzstufe}
	
	\subsection{nMOS-Differenzstufe}

\section{Differenzstufe mit Diodenlast}

\section{x1}

\section{x2}

\section{x3}

-Differenzstufe mit Stromspiegel





% DOCUMENT_END =================================================================
\end{document}