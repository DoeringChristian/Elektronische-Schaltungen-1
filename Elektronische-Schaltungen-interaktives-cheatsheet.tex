% .:: Laden der LaTeX4EI Formelsammlungsvorlage
\documentclass[european]{latex4ei_sheet}

\title{Elektronische \\ Schaltungen \\ Cheatsheet}

\RequirePackage{latex4ei/latex4ei_unicode}

% -- begin custom TOC for circuits
\usepackage{tocloft}

\newcommand{\listcircuitname}{Liste der Schaltungen}
\newlistof{circuit}{cir}{}
\newcommand{\circuit}[2]{%
	\refstepcounter{circuit}
	\section{#1}
	\addcontentsline{cir}{blabla}
	{\parbox{\linewidth}{\includegraphics[width=\linewidth]{#2}\\#1\hfill}\ \\}}
\let\mylistofcircuit\listofcircuit
\renewcommand{\listofcircuit}{\subsection*{\listcircuitname}\vspace{-1.5em}\begin{multicols}{3}\mylistofcircuit\end{multicols}}
% -- end


% DOCUMENT_BEGIN ===============================================================
\begin{document}

\IfFileExists{git.id}{\input{git.id}}{}
\ifdefined\GitRevision\mydate{\GitNiceDate\ (git \GitRevision)}\fi

% Title (needs ./img/Logo.pdf)
\maketitle

\listofcircuit

\circuit{Stromspiegel mit Widerstand}{img/Logo.pdf}
Annahme: Sättigung $\rightarrow V_{\ir GS1} = V_{\ir DD}-RI_{\ir IN}$\\
Überprüfe, ob Annahme korrekt\\
$\frac{I_{\ir out}}{I_{\ir in}} = \frac{(W/L)_2}{(W/L)_1}$
\circuit{Schalter (Transmission-Gate)}{img/Logo.pdf}
$V_{\ir out} \approx V_{\ir in} \rightarrow$ Transistoren in Triodenbereich\\
\circuit{Transistor mit resistiver Last}{img/Logo.pdf}
	\subsection{Source an GND (immer +)}
	%TODO ESB
	Verstärkung: $\frac{V_{\ir out}}{V_{\ir in}} = g_{\ir m1}(r_{\ir DS1}\parallel R_D)$\\
	$r_{\ir out} = r_{\ir DS1}\parallel R_D$\\
	\subsection{$V_{\ir in}$ an Source}
	$r_{\ir in} = R_D \frac{r_{\ir DS}\parallel \frac{1}{g_m}}{r_{\ir DS}\parallel R_D}$
	
	\subsection{$R_S$ an Source}
	
	
\circuit{Sourceschaltung mit Stromquelle (- - -)}{img/Logo.pdf}
%TODO ESB
Verstärkung: $\frac{V_{\ir out}}{V_{\ir in}} = -g_{\ir m1}r_{\ir DS1}$\\
$r_{\ir out} = r_{\ir DS1}$
\circuit{Sourceschaltung mit aktiver Last (- -)}{img/Logo.pdf}
%TODO ESB
Verstärkung: $\frac{V_{\ir out}}{V_{\ir in}} = -g_{\ir m1}(r_{\ir DS1}\parallel r_{\ir DS2})$\\
$r_{\ir out} = r_{\ir DS1}\parallel r_{\ir DS2}$
\section{Sourceschaltung mit Diodenlast}
	\subsection{$V_in$ am unteren Transistor (-)}
	%TODO ESB
	Verstärkung: $\frac{V_{\ir out}}{V_{\ir in}} = -g_{\ir m1}\left( r_{\ir DS1}\parallel r_{\ir DS2}\parallel \frac{1}{g_{\ir m2}}\right)$\\
	$r_{\ir out} = r_{\ir DS1}\parallel r_{\ir DS2}\parallel \frac{1}{g_{\ir m2}}$
	\subsection{$V_in$ am oberen Transistor}

\section{Drainschaltung mit aktiver Last}
	\subsection{ohne Lastkapazität ($\approx 1$)}
	Verstärkung: $\frac{V_{\ir out}}{V_{\ir in}} = \frac{g_{\ir m1}(r_{\ir DS1}\parallel r_{\ir DS2})}{1 + g_{\ir m1}(r_{\ir DS1}\parallel r_{\ir DS2})}$ (kann auch $<1$ sein)\\
	$r_{\ir out} = \frac{r_{\ir DS1}\parallel r_{\ir DS2}}{1 + g_{\ir m1}(r_{\ir DS1}\parallel r_{\ir DS2})}$
	\subsection{mit Lastkapazität}
	%TODO ESB
	Verstärkung: $A(\omega) = \frac{g_{\ir m1}(r_{\ir DS1}\parallel r_{\ir DS2})}{1 + g_{\ir m1}(r_{\ir DS1}\parallel r_{\ir DS2})+\j\omega C_L(r_{\ir DS1}\parallel r_{\ir DS2})}$\\
	Näherung: $\frac{g_{\ir m1}}{g_{\ir m1}+\j\omega C_L}$\\
	Betrag: $\abs{A(\omega)} = \frac{1}{\sqrt{1+(\omega C_L / g_{\ir m1})^2}}$\\
	Phase: $\phi(A(\omega) = \arctan{(-\omega C_L / g_{\ir m1})}$\\

\section{Gateschaltung mit resistiver Last}
%TODO Kleinsignal-ESB
$A \approx g_mR_d$\\
$r_\text{out} = r_{DS} \parallel R_D \approx R_D$\\
$r_\text{in} = R_D\frac{r_{DS}\parallel \frac{1}{g_m}}{r_{DS}\parallel R_D}\approx \frac{1}{g_m}$

\section{Drainschaltung mit aktiver last}
%TODO ESB
$A = \frac{g_{m1}(r_{DS1}\parallel r_{DS2})}{1+r_{DS1}\parallel r_{DS2})} \approx 1$\\
$r_{out}=\frac{r_{DS1}\parallel r_{DS2}}{1+g_m(r_{DS1}\parallel r_{DS2})} \approx \frac{1}{g_{m1}}$

\section{Transistor mit resistiver last}
\section{sourceschaltung mit diodenlast}

\section{Stromspiegel mit Stromquelle}

\section{Differenzstufe mit resistiver Last}

\section{Differenzstufe mit Stromspiegel-Last}
	\subsection{pMOS-Differenzstufe}
	
	\subsection{nMOS-Differenzstufe}

\section{Differenzstufe mit Diodenlast}

\section{x1}

\section{x2}

\section{x3}

-Differenzstufe mit Stromspiegel





% DOCUMENT_END =================================================================
\end{document}