% .:: Laden der LaTeX4EI Formelsammlungsvorlage
\documentclass[european]{latex4ei_sheet}

\title{Elektronische \\ Schaltungen}

\RequirePackage{latex4ei/latex4ei_unicode}

%Packages:

\usepackage[european, lazymos]{circuitikz}
\usepackage{adjustbox}

% DOCUMENT BEGIN
\begin{document}

\IfFileExists{git.id}{\input{git.id}}{}
\ifdefined\GitRevision\mydate{\GitNiceDate\ (git \GitRevision)}\fi

% Title
% ----------------------------------------------------------------------
\maketitle   % requires ./img/Logo.pdf

\section{Transistoren und Passive Bauelemente}


\begin{sectionbox}
    \textbf{\"Ubersicht:}
    \begin{itemize}
        \item Trioden/Linearer -bereich:
            \begin{itemize}
                \item Gatespannung $ V_{GS} > V_ {th} $, Drainspannung $ 0 < V_{DS} < V_{GS} - V_{th} $
                \item gesamter Kanalbereich in (starker) Inversion,
                    an jedem Ort $ y $ im Kanalbereich gilt für die lokale Spannung $ V_{ch}(y) < V_{GS} - V_{th} $
                \item Drainstrom: $ I_D = {W \over L} \mu_n C_{ox} (V_{GS} - V_{th} - {1 \over 2} V_{DS}) V_{DS} $ mit $ C_{ox} = \epsilon_0 \epsilon_r {1 \over t_{ox}} $
                \item Steilheit: $ g_m = {W \over L} \mu_n C_{ox} V_{DS} $, $ g_{DS} = {W \over L} k (V_{GS} - V_{th} - V_{DS}) $
            \end{itemize}
        \item Sättingungsbereich:
            \begin{itemize}
                \item $ V_{GS} > V_{th} $, $ V_{DS} > V_{GS} - V_{th} $ 
                \item Grossteil des Kanalbereiches in Inversion, im drain-nahen Bereich is $ V_{ch}(y) < V_{GS} - V_{th} $ (Inversionsbedingung) jedoch nicht erfüllt
                \item Drainstrom ( $ V_{GS} - V_{th} < v_{DS} $ SH-Modell )
                    $ I_D = I_{D,sat} [1+\lambda {L_{min} \over L} (V_{DS}- V_{DS,sat})] $, $ I_D = {1 \over 2}{W \over L} \mu C_{ox} (V_{GS} - V_{th})^2 [1+\lambda {L_{min} \over L} (V_{DS} - (V_{GS} - V_{th}))] $ 
                \item Steilheit: $ g_m \approx {W \over L} \mu C_{ox} (V_{GS} - V_{th}) $, $ g_{DS} = \lambda {L_{min} \over L} I_D $
            \end{itemize}
        \item Unterschwellbereich:
            \begin{itemize}
                \item $ V_{GS} < V_{th} $
                \item Drainstrom ''sehr klein'' - Inversionsbedingung im Kanalbereich nicht erfüllt.
                \item $ I_D = {W \over L} I_{D,0} \cdot 10^({V_{GS}-V_{th} \over S}) = {W \over L} I_{D,0} \cdot exp({V_{GS}-V_{th} \over S} ln(10)) $
            \end{itemize}
    \end{itemize}
    \subsection{Kleinsignal:}
    \newcommand{\circuitscale}{0.7}
    \begin{minipage}{\textwidth}
        \begin{minipage}{0.3\textwidth}
            Grossignal:\\
            \begin{circuitikz}[scale = \circuitscale, transform shape]
                \ctikzset{tripoles/mos style/arrows}
                \draw 
                % mosfet
                (0,0) node[nmos](mos01){}
                (mos01.B) node[left](){B}
                (mos01.S) node[right](){B}
                (mos01.D) node[right](){D}
                ;
            \end{circuitikz}
        \end{minipage}
        \begin{minipage}{0.5\textwidth}
            Ohne Kapazitäten:\\
            \begin{circuitikz}[scale = \circuitscale, transform shape]
                \ctikzset{tripoles/mos style/arrows}
                \draw 
                % esb
                (3,1) node[label={[font=\footnotesize]above:G}] {} to[short,*-o] (3,1)
                ++(0,-2) node[left](){S} to[short, o-*] ++(2,0)
                to[short, *-*] ++(2,0)
                to[short, *-o] ++(2,0)
                node[right](){S}
                ++(-4,0) to[cisource, f<=$g_m v_{GS}$] ++(0,2)
                to[short, -*] ++(2,0)
                to[R=$g_{DS}$] ++(0,-2)
                ++(0,2) to[short, *-o] ++(2,0)
                node[right](){S}
                ;
            \end{circuitikz}
        \end{minipage}
        \\
        Mit Kapazitäten:
        \\
        \begin{circuitikz}[scale = \circuitscale, transform shape]
            \ctikzset{tripoles/mos style/arrows}
            \draw
            (0,0) node[left](G){G} to[short, o-*] ++(1,0)
            to[C, l_=$C_{GS}+C_{GB}$] ++(0,-2)
            to[short, *-o] ++(-1,0)
            node[left](S){S}
            ++(1,0) to[short, *-*] ++(2,0)
            to[short, *-*] ++(2,0)
            to[short, *-*] ++(2,0)
            to[short, *-o] ++(1,0)
            node[right](S){S}
            ++(-1,0) to[C=$C_{DB}$] ++(0,2)
            to[short, *-o] ++(1,0)
            node[right](D){D}
            ++(-1,0) to[short, *-*] ++(-2,0)
            to[short, *-*] ++(-2,0)
            to[C=$C_{GD}$] ++(-2,0)
            ++(2,-2) to[cisource, f_<=$g_m v_{GS}$] ++(0,2)
            ++(2,0) to[R=$g_{DS}$] ++(0,-2)
            ;
        \end{circuitikz}
    \end{minipage}
\end{sectionbox}

\begin{sectionbox}
    \newcommand{\circuitscale}{0.5}
    \textbf{Grundlegende Schaltungen}
    \subsection{Sourceschaltung n-MOS:}
    \begin{minipage}{0.5\textwidth}
        Grossignal\\
        \begin{circuitikz}[scale = \circuitscale, transform shape]
            \draw
            (0,0) node[left](){$V_{ein}$} to[short, o-*] (0.5,0)
            to[C=$C_{ein}$] ++(2,0)
            to[R, l_=$R_G$] ++(0,-2)
            to[vsource, l=$V_{GS}$] ++(0,-2)
            node[rground](){}
            ++(0,4)
            to[short, *-] ++(0.5,0)
            node[nmos, anchor=B](m1){M1}
            (m1.D) node[circ]{} to[R, l_=$R_D$] ++(0,2)
            node[vcc]{VDD}
            (m1.D) to[C=$C_{aus}$] ++(2,0)
            to[short, *-o] ++(0,0)
            node[right](vaus){$V_{aus}$}
            %to[open, f=$V_{aus}$]
            %++(0,-5) node[ocirc](){} 
            %node[rground](){}
            %++(0,3)
            (m1.S) to[short, -] ++(0,-3.2)
            node[rground](){}
            ;
        \end{circuitikz}
        Kleinsignal\\
        \begin{circuitikz}[scale = \circuitscale, transform shape]
            \draw
            (0,0) node[rground]{} node[ocirc]{}
            ++(0,2) node[left](vein){$V_{ein}$} to[short, o-] ++(1,0)
            to[R=$R_G$] ++(0,-2)
            to[short, -*] ++(2,0)
            node[below, xshift=0.2cm](){S}
            node[circ](s){}
            node[rground]{}
            to[short, *-*] ++(2,0)
            to[short, *-] ++(2,0)
            to[R=$R_{D}$] ++(0,2)
            to[short, *-o] node[above](vaus){$V_{aus}$} ++(1,0)
            ++(0,-2) node[ocirc]{} node[rground]{}
            (s) to[cisource, f<=$g_m v_{ein}$] ++(0,2)
            to[short, -*] ++(2,0)
            node[above](){D}
            node[circ](d){}
            to[R=$r_{DS}$] ++(0,-2)
            (d) to[short, -*] ++(2,0)
            ;
        \end{circuitikz}
    \end{minipage}
    \begin{minipage}{0.5\textwidth}
    \end{minipage}
\end{sectionbox}



% DOCUMENT END
\end{document}
