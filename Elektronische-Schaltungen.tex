% .:: Laden der LaTeX4EI Formelsammlungsvorlage
\documentclass[european]{latex4ei_sheet}

\title{Elektronische \\ Schaltungen}

\RequirePackage{latex4ei/latex4ei_unicode}

% DOCUMENT_BEGIN ===============================================================
\begin{document}

\IfFileExists{git.id}{\input{git.id}}{}
\ifdefined\GitRevision\mydate{\GitNiceDate\ (git \GitRevision)}\fi

% Title (needs ./img/Logo.pdf)
\maketitle

\section{Transistoren und passive Bauteile}
\subsection{MOS-Transistoren}

\begin{sectionbox}
	\subsubsection{Arbeitsbereiche} %TODO I_D Formeln mit Kanallängenmodulationsfaktor
	Ein MOS-Transistor hat drei verschiedene Arbeitsbereiche. Falls Gate und Drain verbunden sind befindet er sich im Triodenbereich bzw. Unterschwellbereich.
	\begin{tablebox}{ll}
	& Triodenbereich \\
	\cmrule
	Voraussetzung & $V_{\ir GS} > V_{\ir th}$ und $V_{\ir DS} < V_{\ir GS} - V_{\ir th}$ \\
	Drainstrom & $I_{\ir D} = \frac{W}{L}k(V_{\ir GS} - V_{\ir th} - \frac{1}{2} V_{\ir DS})V_{\ir DS}$ \\
	Transkonduktanz & $g_m = \frac{\diff I}{\diff V_{\ir GS}}|_{V_{\ir DS} =\ir const.} = \frac{W}{L}kV_{\ir DS}$ \\
	Diff. Ausgangsleitw. & $g_{\ir DS} = \frac{W}{L}k(V_{\ir GS}-V_{\ir th}-V_{\ir DS})$\\
	Charakteristiken & (TODO)\\
	%TODO Bilder der Ein- und Ausgangskennlinie einfügen VL2: F22
	\end{tablebox}
	\begin{tablebox}{ll}
	& Sättigungsbereich \\
	\cmrule
	Voraussetzung & $V_{\ir GS} > V_{\ir th}$ und $V_{\ir DS} > V_{\ir GS} - V_{\ir th}$ \\
	Drainstrom & $I_{\ir D} = \frac{W}{2L}k(V_{\ir GS} - V_{\ir th})^2$ \\
	Transkonduktanz & $g_m = \frac{\diff I}{\diff V_{\ir GS}}|_{V_{\ir DS} =\ir const.} = \frac{W}{L}k(V_{\ir GS}-V_{\ir th})$ \\
	Diff. Ausgangsleitw. & $g_{\ir DS} = \frac{W}{2L}k(V_{\ir GS}-V_{\ir th})^2\lambda \frac{L_{\ir min}}{L}$ \\
	Charakteristiken & (TODO)\\
	%TODO Bilder der Ein- und Ausgangskennlinie einfügen VL2: F28
	\end{tablebox}
	\begin{tablebox}{ll}
	& Unterschwellbereich \\
	\cmrule
	Voraussetzung & $V_{\ir GS} < V_{\ir th}$\\
	Drainstrom & $I_{\ir D} = \frac{W}{L}I_{\ir D,0}10^{\frac{V_{\ir GS}-V_{\ir th}}{S}}\left[ 1-\exp{\left( -\frac{V_{\ir DS}q}{kT}\right)}\right]$ \\
	Charakteristiken & (TODO)\\
	%TODO Bilder der Ein- und Ausgangskennlinie einfügen VL3: F6
	\end{tablebox}
	S: inverse Unterschwellstrom-Steigung\\
	$k = C_{\ir ox}\mu_{\ir n}$\\
	$\lambda$: Kanallängenreduktionsfaltor (prozessabhängig)
\end{sectionbox}
\begin{sectionbox}
	\subsubsection{Kleinsignalparameter}
	Modelle gut für Handentwurf von Schaltungen durch geringe Komplexität und richtiger Abbildung von elementaren physikalischen Gesetzmäßigkeiten.\\
	\begin{itemize}
		\item Gleichspannungsquellen werden zu Kurzschlüssen
		\item Gleichstromquellen werden zu Leerläufen
	\end{itemize}
	Bei Vernachlässigung des Substrateffektes und bei Vernachlässigung der Transistorkapazitäten (niedrige Frequenzen)\\
	(TODO)\\
	%TODO Kleinsignal-ESB VL3: F14
	Bei Vernachlässigung des Substrateffektes aber mit Berücksichtigung der Transistorkapazitäten (höhere Frequenzen)\\
	(TODO)\\
	%TODO Kleinsignal-ESB VL3: F15
	\textbf{Probleme:}
	\begin{enumerate}
		\item Übergang des AP von Unterschwellbereich in Inversionsbereich (Unstetigkeit Strom und Kleinsignalparam.)
		\item Übergang von Triodenbereich in Sättigungsbereich (Unstetigkeit Kleinsignalparam.)
		\item AP-unabhängigkeit nur mäßig korrekt wiedergegeben
	\end{enumerate}
\end{sectionbox}
\begin{sectionbox}
	\subsubsection{Kennlinien realer Bauteile}
	Bestimmung der Schwellenwertspannung aus Eingangskennlinie:\\
	%TODO Bild VL3: F22
\end{sectionbox}

\section{Transistorschaltungsentwurf}

\begin{sectionbox}
	\subsection{Sourceschaltung}
	\subsubsection{niedrige Frequenzen}
	%TODO Bilder VL3: 8-9
	Verstärkung: $\frac{V_{\ir out}}{V_{\ir in}} = -g_{\ir m}(r_{\ir DS}\parallel R_{\ir D})$\\
	Eingangswiderstand: $r_{\ir in} = \frac{V_{\ir in}}{I_{\ir in}} = R_{\ir G}$\\
	Ausgangswiderstand: $r_{\ir out} = \frac{V_{\ir out}}{I_{\ir out}}|_{v_{\ir in} = 0} = r_{\ir DS}\parallel R_D \approx R_D \quad (R_D \ll r_{\ir DS})$\\
	\subsubsection{höhere Frequenzen}
	Komplexer Eingangsleitwert: $y_{\ir in} = \frac{1}{R_{\ir G}}+\j\omega[C_{\ir GS} + C_{\ir GD}(1-A)]$
	Der \textbf{Miller-Effekt} gewichtet die Kapazität zwischen Eingang und Ausgang der Schaltung $C_{\ir GD}$ wird mit Faktor $1+\abs{A}$ gewichtet.
	%TODO ESB VL3: 18	
	\begin{itemize}
		\item Schaltung mit inventierenden Verstärkerstufen: Bewirkt Bandbegrenzung (Nachteil!!)
		\item große wirksame Kapazitäten aus kleinen Kapazitäten erzeugen (Miller-Integrator)
		\item Es gibt Schaltungen, die den Miller-Effekt verringern
	\end{itemize}
\end{sectionbox}
\begin{sectionbox}
	\subsection{Drainschaltung}
	\subsubsection{niedrige Frequenzen}
		%TODO ESB VL3: 21
		Verstärkung: $\frac{V_{\ir out}}{V_{\ir in}} = \frac{g_m}{\left(g_m+\frac{1}{r_{\ir DS}}+\frac{1}{R_{\ir D}}\right)} \approx 1 \left(g_m\gg\frac{1}{R_S}, \frac{1}{r_{\ir DS}}\right)$\\
		Eingangswiderstand: $r_{\ir in} = \frac{V_{\ir in}}{I_{\ir in}} = R_{\ir G}$\\
		Ausgangswiderstand: $r_{\ir out} = \frac{V_{\ir out}}{I_{\ir out}}|_{v_{\ir in} = 0} = R_S\parallel r_{\ir DS}\parallel \frac{1}{g_m} \approx \frac{1}{g_m} \quad (g_m\ll \frac{1}{R_S}, \frac{1}{r_{\ir DS}})$
	\subsubsection{höhere Frequenzen}
	%TODO ESB VL3: 28
\end{sectionbox}
\begin{sectionbox}
	\subsection{Gateschaltung}
	\subsubsection{niedrige Frequenzen}
	%TODO ESB VL3: 30	
	Verstärkung: $\frac{V_{\ir out}}{V_{\ir in}} \approx g_m(R_D\parallel r_{\ir DS})$\\
	Eingangswiderstand: $r_{\ir in} \approx \frac{1}{g_m}$\\
	Ausgangswiderstand: $r_{\ir out} \approx R_D$\\
	\subsubsection{höhere Frequenzen}
	%TODO ESB VL3: 32
	\subsection{Kleinsignal ESB von komplementären Bauteilen}
	%TODO ESB VL3: 35
	\subsection{Bandweite von Schaltungen}
	Verstärkung: $A(\omega) = \frac{V_{\ir out}}{V_{\ir in}}$\\
	Betrag: $\abs{A(\omega)} = \sqrt{A(\omega)^2}$\\
	Phase: $\varphi(A(\omega)) = \arctan\left(\frac{\Im{A(\omega)}}{\Re{A(\omega)}}\right)$\\
	Verstärkungs-Bandbreite-Produkt: $\abs{A(2\pi f_0)} = 1$\\
\end{sectionbox}
\section{Analoge Schaltungen}
\begin{sectionbox}
	\subsection{Invertierende MOS-Verstärkerschaltungen}
	Lösungsansätze (TODO) %TODO	
	\subsection{Source-Folger}
	Transistor ist notwendigerweise immer in Sättigung\\
	Anwendungen: Impedanzwandlung, Gleichspannungs-Pegelanpassung\\
	Verstärkung: $\frac{V_{\ir out}}{V_{\ir in}} = \frac{g_m}{g_m+g_{\ir DS}}\approx 1$\\
	Ausgangswiderstand: $r_{\ir out} = \frac{1}{g_m+g_{\ir DS}}\approx \frac{1}{g_m}$
	%TODO ESB VL5: 30
	\subsection{Stromspiegel}
	Stromverstärkung: $I_{\ir D2} = \frac{W_2}{W_1}\frac{L_1}{L_2}I_{\ir D1}$\\
	Eingangswiderstand: $r_{\ir in} \approx \frac{1}{g_m}$\\
	Ausgangswiderstand: $r_{\ir out} \approx \frac{1}{g_{\ir DS2}}$\\
	%TODO ESB VL5: 35 
	\subsection{Stromquellenschaltung}
	%TODO VL 5: 39f.
	\subsection{Differenzstufe} 
	%TODO Gleichtakt-ESB VL5: 46
	Es wird eine Verstärkung des Differenzsignals unabhängig vom Gleichtaktsignal angestrebt (d.h. hohe Gleichtaktunterdrückung)
	Gleichtaktunterdrückung: $\text{CMRR} = \frac{\text{Differenzverstärkung}}{\text{Gleichtaktverstärkung}}$	
\end{sectionbox}
% DOCUMENT_END =================================================================
\end{document}