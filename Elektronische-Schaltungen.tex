% .:: Laden der LaTeX4EI Formelsammlungsvorlage
\documentclass[european]{latex4ei_sheet}

\title{Elektronische \\ Schaltungen}

\RequirePackage{latex4ei/latex4ei_unicode}

% DOCUMENT_BEGIN ===============================================================
\begin{document}

\IfFileExists{git.id}{\input{git.id}}{}
\ifdefined\GitRevision\mydate{\GitNiceDate\ (git \GitRevision)}\fi

% Title (needs ./img/Logo.pdf)
\maketitle

\section{Transistoren und passive Bauteile}
\subsection{MOS-Transistoren}

\begin{sectionbox}
	\subsubsection{Arbeitsbereiche}
	Ein MOS-Transistor hat drei verschiedene Arbeitsbereiche
	\begin{tablebox}{ll}
	& Triodenbereich \\
	\cmrule
	Voraussetzung & $V_{\ir GS} > V_{\ir th}$ und $V_{\ir DS} < V_{\ir GS} - V_{\ir th}$ \\
	Drainstrom & $I_{\ir D} = \frac{W}{L}k(V_{\ir GS} - V_{\ir th} - \frac{1}{2} V_{\ir DS})V_{\ir DS}$ \\
	Transkonduktanz & $g_m = \frac{\diff I}{\diff V_{\ir GS}}|_{V_{\ir DS} =\ir const.} = \frac{W}{L}kV_{\ir DS}$ \\
	Diff. Ausgangsleitw. & $g_{\ir DS} = \frac{W}{L}k(V_{\ir GS}-V_{\ir th}-V_{\ir DS})$\\
	Charakteristiken & (TODO)\\
	%TODO Bilder der Ein- und Ausgangskennlinie einfügen VL2: F22
	\end{tablebox}
	\begin{tablebox}{ll}
	& Sättigungsbereich \\
	\cmrule
	Voraussetzung & $V_{\ir GS} > V_{\ir th}$ und $V_{\ir DS} > V_{\ir GS} - V_{\ir th}$ \\
	Drainstrom & $I_{\ir D} = \frac{W}{2L}k(V_{\ir GS} - V_{\ir th})^2$ \\
	Transkonduktanz & $g_m = \frac{\diff I}{\diff V_{\ir GS}}|_{V_{\ir DS} =\ir const.} = \frac{W}{L}k(V_{\ir GS}-V_{\ir th})$ \\
	Diff. Ausgangsleitw. & $g_{\ir DS} = \frac{W}{2L}k(V_{\ir GS}-V_{\ir th})^2\lambda \frac{L_{\ir min}}{L}$ \\
	Charakteristiken & (TODO)\\
	%TODO Bilder der Ein- und Ausgangskennlinie einfügen VL2: F28
	\end{tablebox}
	\begin{tablebox}{ll}
	& Unterschwellbereich \\
	\cmrule
	Voraussetzung & $V_{\ir GS} < V_{\ir th}$\\
	Drainstrom & $I_{\ir D} = \frac{W}{L}I_{\ir D,0}10^{\frac{V_{\ir GS}-V_{\ir th}}{S}}\left[ 1-\exp{\left( -\frac{V_{\ir DS}q}{kT}\right)}\right]$ \\
	Charakteristiken & (TODO)\\
	%TODO Bilder der Ein- und Ausgangskennlinie einfügen VL3: F6
	\end{tablebox}
	S: inverse Unterschwellstrom-Steigung\\
	$k = C_{\ir ox}\mu_{\ir n}$\\
	$\lambda$: Kanallängenreduktionsfaltor (prozessabhängig)
\end{sectionbox}
\begin{sectionbox}
	\subsubsection{Kleinsignalparameter}
	Modelle gut für Handentwurf von Schaltungen durch geringe Komplexität und richtiger Abbildung von elementaren physikalischen Gesetzmäßigkeiten.\\
	Bei Vernachlässigung des Substrateffektes und bei Vernachlässigung der Transistorkapazitäten (niedrige Frequenzen)\\
	(TODO)\\
	%TODO Kleinsignal-ESB VL3: F14
	Bei Vernachlässigung des Substrateffektes aber mit Berücksichtigung der Transistorkapazitäten (höhere Frequenzen)\\
	(TODO)\\
	%TODO Kleinsignal-ESB VL3: F15
	\textbf{Probleme:}
	\begin{enumerate}
		\item Übergang des AP von Unterschwellbereich in Inversionsbereich (Unstetigkeit Strom und Kleinsignalparam.)
		\item Übergang von Triodenbereich in Sättigungsbereich (Unstetigkeit Kleinsignalparam.)
		\item AP-unabhängigkeit nur mäßig korrekt wiedergegeben
	\end{enumerate}
\end{sectionbox}
\begin{sectionbox}
	\subsubsection{Kennlinien realer Bauteile}
	Bestimmung der Schwellenwertspannung aus Eingangskennlinie:\\
	%TODO Bild VL3: F22
\end{sectionbox}
	%
% DOCUMENT_END =================================================================
\end{document}